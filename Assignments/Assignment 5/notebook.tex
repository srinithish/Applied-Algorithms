
% Default to the notebook output style

    


% Inherit from the specified cell style.




    
\documentclass[11pt]{article}

    
    
    \usepackage[T1]{fontenc}
    % Nicer default font (+ math font) than Computer Modern for most use cases
    \usepackage{mathpazo}

    % Basic figure setup, for now with no caption control since it's done
    % automatically by Pandoc (which extracts ![](path) syntax from Markdown).
    \usepackage{graphicx}
    % We will generate all images so they have a width \maxwidth. This means
    % that they will get their normal width if they fit onto the page, but
    % are scaled down if they would overflow the margins.
    \makeatletter
    \def\maxwidth{\ifdim\Gin@nat@width>\linewidth\linewidth
    \else\Gin@nat@width\fi}
    \makeatother
    \let\Oldincludegraphics\includegraphics
    % Set max figure width to be 80% of text width, for now hardcoded.
    \renewcommand{\includegraphics}[1]{\Oldincludegraphics[width=.8\maxwidth]{#1}}
    % Ensure that by default, figures have no caption (until we provide a
    % proper Figure object with a Caption API and a way to capture that
    % in the conversion process - todo).
    \usepackage{caption}
    \DeclareCaptionLabelFormat{nolabel}{}
    \captionsetup{labelformat=nolabel}

    \usepackage{adjustbox} % Used to constrain images to a maximum size 
    \usepackage{xcolor} % Allow colors to be defined
    \usepackage{enumerate} % Needed for markdown enumerations to work
    \usepackage{geometry} % Used to adjust the document margins
    \usepackage{amsmath} % Equations
    \usepackage{amssymb} % Equations
    \usepackage{textcomp} % defines textquotesingle
    % Hack from http://tex.stackexchange.com/a/47451/13684:
    \AtBeginDocument{%
        \def\PYZsq{\textquotesingle}% Upright quotes in Pygmentized code
    }
    \usepackage{upquote} % Upright quotes for verbatim code
    \usepackage{eurosym} % defines \euro
    \usepackage[mathletters]{ucs} % Extended unicode (utf-8) support
    \usepackage[utf8x]{inputenc} % Allow utf-8 characters in the tex document
    \usepackage{fancyvrb} % verbatim replacement that allows latex
    \usepackage{grffile} % extends the file name processing of package graphics 
                         % to support a larger range 
    % The hyperref package gives us a pdf with properly built
    % internal navigation ('pdf bookmarks' for the table of contents,
    % internal cross-reference links, web links for URLs, etc.)
    \usepackage{hyperref}
    \usepackage{longtable} % longtable support required by pandoc >1.10
    \usepackage{booktabs}  % table support for pandoc > 1.12.2
    \usepackage[inline]{enumitem} % IRkernel/repr support (it uses the enumerate* environment)
    \usepackage[normalem]{ulem} % ulem is needed to support strikethroughs (\sout)
                                % normalem makes italics be italics, not underlines
    

    
    
    % Colors for the hyperref package
    \definecolor{urlcolor}{rgb}{0,.145,.698}
    \definecolor{linkcolor}{rgb}{.71,0.21,0.01}
    \definecolor{citecolor}{rgb}{.12,.54,.11}

    % ANSI colors
    \definecolor{ansi-black}{HTML}{3E424D}
    \definecolor{ansi-black-intense}{HTML}{282C36}
    \definecolor{ansi-red}{HTML}{E75C58}
    \definecolor{ansi-red-intense}{HTML}{B22B31}
    \definecolor{ansi-green}{HTML}{00A250}
    \definecolor{ansi-green-intense}{HTML}{007427}
    \definecolor{ansi-yellow}{HTML}{DDB62B}
    \definecolor{ansi-yellow-intense}{HTML}{B27D12}
    \definecolor{ansi-blue}{HTML}{208FFB}
    \definecolor{ansi-blue-intense}{HTML}{0065CA}
    \definecolor{ansi-magenta}{HTML}{D160C4}
    \definecolor{ansi-magenta-intense}{HTML}{A03196}
    \definecolor{ansi-cyan}{HTML}{60C6C8}
    \definecolor{ansi-cyan-intense}{HTML}{258F8F}
    \definecolor{ansi-white}{HTML}{C5C1B4}
    \definecolor{ansi-white-intense}{HTML}{A1A6B2}

    % commands and environments needed by pandoc snippets
    % extracted from the output of `pandoc -s`
    \providecommand{\tightlist}{%
      \setlength{\itemsep}{0pt}\setlength{\parskip}{0pt}}
    \DefineVerbatimEnvironment{Highlighting}{Verbatim}{commandchars=\\\{\}}
    % Add ',fontsize=\small' for more characters per line
    \newenvironment{Shaded}{}{}
    \newcommand{\KeywordTok}[1]{\textcolor[rgb]{0.00,0.44,0.13}{\textbf{{#1}}}}
    \newcommand{\DataTypeTok}[1]{\textcolor[rgb]{0.56,0.13,0.00}{{#1}}}
    \newcommand{\DecValTok}[1]{\textcolor[rgb]{0.25,0.63,0.44}{{#1}}}
    \newcommand{\BaseNTok}[1]{\textcolor[rgb]{0.25,0.63,0.44}{{#1}}}
    \newcommand{\FloatTok}[1]{\textcolor[rgb]{0.25,0.63,0.44}{{#1}}}
    \newcommand{\CharTok}[1]{\textcolor[rgb]{0.25,0.44,0.63}{{#1}}}
    \newcommand{\StringTok}[1]{\textcolor[rgb]{0.25,0.44,0.63}{{#1}}}
    \newcommand{\CommentTok}[1]{\textcolor[rgb]{0.38,0.63,0.69}{\textit{{#1}}}}
    \newcommand{\OtherTok}[1]{\textcolor[rgb]{0.00,0.44,0.13}{{#1}}}
    \newcommand{\AlertTok}[1]{\textcolor[rgb]{1.00,0.00,0.00}{\textbf{{#1}}}}
    \newcommand{\FunctionTok}[1]{\textcolor[rgb]{0.02,0.16,0.49}{{#1}}}
    \newcommand{\RegionMarkerTok}[1]{{#1}}
    \newcommand{\ErrorTok}[1]{\textcolor[rgb]{1.00,0.00,0.00}{\textbf{{#1}}}}
    \newcommand{\NormalTok}[1]{{#1}}
    
    % Additional commands for more recent versions of Pandoc
    \newcommand{\ConstantTok}[1]{\textcolor[rgb]{0.53,0.00,0.00}{{#1}}}
    \newcommand{\SpecialCharTok}[1]{\textcolor[rgb]{0.25,0.44,0.63}{{#1}}}
    \newcommand{\VerbatimStringTok}[1]{\textcolor[rgb]{0.25,0.44,0.63}{{#1}}}
    \newcommand{\SpecialStringTok}[1]{\textcolor[rgb]{0.73,0.40,0.53}{{#1}}}
    \newcommand{\ImportTok}[1]{{#1}}
    \newcommand{\DocumentationTok}[1]{\textcolor[rgb]{0.73,0.13,0.13}{\textit{{#1}}}}
    \newcommand{\AnnotationTok}[1]{\textcolor[rgb]{0.38,0.63,0.69}{\textbf{\textit{{#1}}}}}
    \newcommand{\CommentVarTok}[1]{\textcolor[rgb]{0.38,0.63,0.69}{\textbf{\textit{{#1}}}}}
    \newcommand{\VariableTok}[1]{\textcolor[rgb]{0.10,0.09,0.49}{{#1}}}
    \newcommand{\ControlFlowTok}[1]{\textcolor[rgb]{0.00,0.44,0.13}{\textbf{{#1}}}}
    \newcommand{\OperatorTok}[1]{\textcolor[rgb]{0.40,0.40,0.40}{{#1}}}
    \newcommand{\BuiltInTok}[1]{{#1}}
    \newcommand{\ExtensionTok}[1]{{#1}}
    \newcommand{\PreprocessorTok}[1]{\textcolor[rgb]{0.74,0.48,0.00}{{#1}}}
    \newcommand{\AttributeTok}[1]{\textcolor[rgb]{0.49,0.56,0.16}{{#1}}}
    \newcommand{\InformationTok}[1]{\textcolor[rgb]{0.38,0.63,0.69}{\textbf{\textit{{#1}}}}}
    \newcommand{\WarningTok}[1]{\textcolor[rgb]{0.38,0.63,0.69}{\textbf{\textit{{#1}}}}}
    
    
    % Define a nice break command that doesn't care if a line doesn't already
    % exist.
    \def\br{\hspace*{\fill} \\* }
    % Math Jax compatability definitions
    \def\gt{>}
    \def\lt{<}
    % Document parameters
    \title{Applied\_algos\_Assignment\_5}
    
    
    

    % Pygments definitions
    
\makeatletter
\def\PY@reset{\let\PY@it=\relax \let\PY@bf=\relax%
    \let\PY@ul=\relax \let\PY@tc=\relax%
    \let\PY@bc=\relax \let\PY@ff=\relax}
\def\PY@tok#1{\csname PY@tok@#1\endcsname}
\def\PY@toks#1+{\ifx\relax#1\empty\else%
    \PY@tok{#1}\expandafter\PY@toks\fi}
\def\PY@do#1{\PY@bc{\PY@tc{\PY@ul{%
    \PY@it{\PY@bf{\PY@ff{#1}}}}}}}
\def\PY#1#2{\PY@reset\PY@toks#1+\relax+\PY@do{#2}}

\expandafter\def\csname PY@tok@w\endcsname{\def\PY@tc##1{\textcolor[rgb]{0.73,0.73,0.73}{##1}}}
\expandafter\def\csname PY@tok@c\endcsname{\let\PY@it=\textit\def\PY@tc##1{\textcolor[rgb]{0.25,0.50,0.50}{##1}}}
\expandafter\def\csname PY@tok@cp\endcsname{\def\PY@tc##1{\textcolor[rgb]{0.74,0.48,0.00}{##1}}}
\expandafter\def\csname PY@tok@k\endcsname{\let\PY@bf=\textbf\def\PY@tc##1{\textcolor[rgb]{0.00,0.50,0.00}{##1}}}
\expandafter\def\csname PY@tok@kp\endcsname{\def\PY@tc##1{\textcolor[rgb]{0.00,0.50,0.00}{##1}}}
\expandafter\def\csname PY@tok@kt\endcsname{\def\PY@tc##1{\textcolor[rgb]{0.69,0.00,0.25}{##1}}}
\expandafter\def\csname PY@tok@o\endcsname{\def\PY@tc##1{\textcolor[rgb]{0.40,0.40,0.40}{##1}}}
\expandafter\def\csname PY@tok@ow\endcsname{\let\PY@bf=\textbf\def\PY@tc##1{\textcolor[rgb]{0.67,0.13,1.00}{##1}}}
\expandafter\def\csname PY@tok@nb\endcsname{\def\PY@tc##1{\textcolor[rgb]{0.00,0.50,0.00}{##1}}}
\expandafter\def\csname PY@tok@nf\endcsname{\def\PY@tc##1{\textcolor[rgb]{0.00,0.00,1.00}{##1}}}
\expandafter\def\csname PY@tok@nc\endcsname{\let\PY@bf=\textbf\def\PY@tc##1{\textcolor[rgb]{0.00,0.00,1.00}{##1}}}
\expandafter\def\csname PY@tok@nn\endcsname{\let\PY@bf=\textbf\def\PY@tc##1{\textcolor[rgb]{0.00,0.00,1.00}{##1}}}
\expandafter\def\csname PY@tok@ne\endcsname{\let\PY@bf=\textbf\def\PY@tc##1{\textcolor[rgb]{0.82,0.25,0.23}{##1}}}
\expandafter\def\csname PY@tok@nv\endcsname{\def\PY@tc##1{\textcolor[rgb]{0.10,0.09,0.49}{##1}}}
\expandafter\def\csname PY@tok@no\endcsname{\def\PY@tc##1{\textcolor[rgb]{0.53,0.00,0.00}{##1}}}
\expandafter\def\csname PY@tok@nl\endcsname{\def\PY@tc##1{\textcolor[rgb]{0.63,0.63,0.00}{##1}}}
\expandafter\def\csname PY@tok@ni\endcsname{\let\PY@bf=\textbf\def\PY@tc##1{\textcolor[rgb]{0.60,0.60,0.60}{##1}}}
\expandafter\def\csname PY@tok@na\endcsname{\def\PY@tc##1{\textcolor[rgb]{0.49,0.56,0.16}{##1}}}
\expandafter\def\csname PY@tok@nt\endcsname{\let\PY@bf=\textbf\def\PY@tc##1{\textcolor[rgb]{0.00,0.50,0.00}{##1}}}
\expandafter\def\csname PY@tok@nd\endcsname{\def\PY@tc##1{\textcolor[rgb]{0.67,0.13,1.00}{##1}}}
\expandafter\def\csname PY@tok@s\endcsname{\def\PY@tc##1{\textcolor[rgb]{0.73,0.13,0.13}{##1}}}
\expandafter\def\csname PY@tok@sd\endcsname{\let\PY@it=\textit\def\PY@tc##1{\textcolor[rgb]{0.73,0.13,0.13}{##1}}}
\expandafter\def\csname PY@tok@si\endcsname{\let\PY@bf=\textbf\def\PY@tc##1{\textcolor[rgb]{0.73,0.40,0.53}{##1}}}
\expandafter\def\csname PY@tok@se\endcsname{\let\PY@bf=\textbf\def\PY@tc##1{\textcolor[rgb]{0.73,0.40,0.13}{##1}}}
\expandafter\def\csname PY@tok@sr\endcsname{\def\PY@tc##1{\textcolor[rgb]{0.73,0.40,0.53}{##1}}}
\expandafter\def\csname PY@tok@ss\endcsname{\def\PY@tc##1{\textcolor[rgb]{0.10,0.09,0.49}{##1}}}
\expandafter\def\csname PY@tok@sx\endcsname{\def\PY@tc##1{\textcolor[rgb]{0.00,0.50,0.00}{##1}}}
\expandafter\def\csname PY@tok@m\endcsname{\def\PY@tc##1{\textcolor[rgb]{0.40,0.40,0.40}{##1}}}
\expandafter\def\csname PY@tok@gh\endcsname{\let\PY@bf=\textbf\def\PY@tc##1{\textcolor[rgb]{0.00,0.00,0.50}{##1}}}
\expandafter\def\csname PY@tok@gu\endcsname{\let\PY@bf=\textbf\def\PY@tc##1{\textcolor[rgb]{0.50,0.00,0.50}{##1}}}
\expandafter\def\csname PY@tok@gd\endcsname{\def\PY@tc##1{\textcolor[rgb]{0.63,0.00,0.00}{##1}}}
\expandafter\def\csname PY@tok@gi\endcsname{\def\PY@tc##1{\textcolor[rgb]{0.00,0.63,0.00}{##1}}}
\expandafter\def\csname PY@tok@gr\endcsname{\def\PY@tc##1{\textcolor[rgb]{1.00,0.00,0.00}{##1}}}
\expandafter\def\csname PY@tok@ge\endcsname{\let\PY@it=\textit}
\expandafter\def\csname PY@tok@gs\endcsname{\let\PY@bf=\textbf}
\expandafter\def\csname PY@tok@gp\endcsname{\let\PY@bf=\textbf\def\PY@tc##1{\textcolor[rgb]{0.00,0.00,0.50}{##1}}}
\expandafter\def\csname PY@tok@go\endcsname{\def\PY@tc##1{\textcolor[rgb]{0.53,0.53,0.53}{##1}}}
\expandafter\def\csname PY@tok@gt\endcsname{\def\PY@tc##1{\textcolor[rgb]{0.00,0.27,0.87}{##1}}}
\expandafter\def\csname PY@tok@err\endcsname{\def\PY@bc##1{\setlength{\fboxsep}{0pt}\fcolorbox[rgb]{1.00,0.00,0.00}{1,1,1}{\strut ##1}}}
\expandafter\def\csname PY@tok@kc\endcsname{\let\PY@bf=\textbf\def\PY@tc##1{\textcolor[rgb]{0.00,0.50,0.00}{##1}}}
\expandafter\def\csname PY@tok@kd\endcsname{\let\PY@bf=\textbf\def\PY@tc##1{\textcolor[rgb]{0.00,0.50,0.00}{##1}}}
\expandafter\def\csname PY@tok@kn\endcsname{\let\PY@bf=\textbf\def\PY@tc##1{\textcolor[rgb]{0.00,0.50,0.00}{##1}}}
\expandafter\def\csname PY@tok@kr\endcsname{\let\PY@bf=\textbf\def\PY@tc##1{\textcolor[rgb]{0.00,0.50,0.00}{##1}}}
\expandafter\def\csname PY@tok@bp\endcsname{\def\PY@tc##1{\textcolor[rgb]{0.00,0.50,0.00}{##1}}}
\expandafter\def\csname PY@tok@fm\endcsname{\def\PY@tc##1{\textcolor[rgb]{0.00,0.00,1.00}{##1}}}
\expandafter\def\csname PY@tok@vc\endcsname{\def\PY@tc##1{\textcolor[rgb]{0.10,0.09,0.49}{##1}}}
\expandafter\def\csname PY@tok@vg\endcsname{\def\PY@tc##1{\textcolor[rgb]{0.10,0.09,0.49}{##1}}}
\expandafter\def\csname PY@tok@vi\endcsname{\def\PY@tc##1{\textcolor[rgb]{0.10,0.09,0.49}{##1}}}
\expandafter\def\csname PY@tok@vm\endcsname{\def\PY@tc##1{\textcolor[rgb]{0.10,0.09,0.49}{##1}}}
\expandafter\def\csname PY@tok@sa\endcsname{\def\PY@tc##1{\textcolor[rgb]{0.73,0.13,0.13}{##1}}}
\expandafter\def\csname PY@tok@sb\endcsname{\def\PY@tc##1{\textcolor[rgb]{0.73,0.13,0.13}{##1}}}
\expandafter\def\csname PY@tok@sc\endcsname{\def\PY@tc##1{\textcolor[rgb]{0.73,0.13,0.13}{##1}}}
\expandafter\def\csname PY@tok@dl\endcsname{\def\PY@tc##1{\textcolor[rgb]{0.73,0.13,0.13}{##1}}}
\expandafter\def\csname PY@tok@s2\endcsname{\def\PY@tc##1{\textcolor[rgb]{0.73,0.13,0.13}{##1}}}
\expandafter\def\csname PY@tok@sh\endcsname{\def\PY@tc##1{\textcolor[rgb]{0.73,0.13,0.13}{##1}}}
\expandafter\def\csname PY@tok@s1\endcsname{\def\PY@tc##1{\textcolor[rgb]{0.73,0.13,0.13}{##1}}}
\expandafter\def\csname PY@tok@mb\endcsname{\def\PY@tc##1{\textcolor[rgb]{0.40,0.40,0.40}{##1}}}
\expandafter\def\csname PY@tok@mf\endcsname{\def\PY@tc##1{\textcolor[rgb]{0.40,0.40,0.40}{##1}}}
\expandafter\def\csname PY@tok@mh\endcsname{\def\PY@tc##1{\textcolor[rgb]{0.40,0.40,0.40}{##1}}}
\expandafter\def\csname PY@tok@mi\endcsname{\def\PY@tc##1{\textcolor[rgb]{0.40,0.40,0.40}{##1}}}
\expandafter\def\csname PY@tok@il\endcsname{\def\PY@tc##1{\textcolor[rgb]{0.40,0.40,0.40}{##1}}}
\expandafter\def\csname PY@tok@mo\endcsname{\def\PY@tc##1{\textcolor[rgb]{0.40,0.40,0.40}{##1}}}
\expandafter\def\csname PY@tok@ch\endcsname{\let\PY@it=\textit\def\PY@tc##1{\textcolor[rgb]{0.25,0.50,0.50}{##1}}}
\expandafter\def\csname PY@tok@cm\endcsname{\let\PY@it=\textit\def\PY@tc##1{\textcolor[rgb]{0.25,0.50,0.50}{##1}}}
\expandafter\def\csname PY@tok@cpf\endcsname{\let\PY@it=\textit\def\PY@tc##1{\textcolor[rgb]{0.25,0.50,0.50}{##1}}}
\expandafter\def\csname PY@tok@c1\endcsname{\let\PY@it=\textit\def\PY@tc##1{\textcolor[rgb]{0.25,0.50,0.50}{##1}}}
\expandafter\def\csname PY@tok@cs\endcsname{\let\PY@it=\textit\def\PY@tc##1{\textcolor[rgb]{0.25,0.50,0.50}{##1}}}

\def\PYZbs{\char`\\}
\def\PYZus{\char`\_}
\def\PYZob{\char`\{}
\def\PYZcb{\char`\}}
\def\PYZca{\char`\^}
\def\PYZam{\char`\&}
\def\PYZlt{\char`\<}
\def\PYZgt{\char`\>}
\def\PYZsh{\char`\#}
\def\PYZpc{\char`\%}
\def\PYZdl{\char`\$}
\def\PYZhy{\char`\-}
\def\PYZsq{\char`\'}
\def\PYZdq{\char`\"}
\def\PYZti{\char`\~}
% for compatibility with earlier versions
\def\PYZat{@}
\def\PYZlb{[}
\def\PYZrb{]}
\makeatother


    % Exact colors from NB
    \definecolor{incolor}{rgb}{0.0, 0.0, 0.5}
    \definecolor{outcolor}{rgb}{0.545, 0.0, 0.0}



    
    % Prevent overflowing lines due to hard-to-break entities
    \sloppy 
    % Setup hyperref package
    \hypersetup{
      breaklinks=true,  % so long urls are correctly broken across lines
      colorlinks=true,
      urlcolor=urlcolor,
      linkcolor=linkcolor,
      citecolor=citecolor,
      }
    % Slightly bigger margins than the latex defaults
    
    \geometry{verbose,tmargin=1in,bmargin=1in,lmargin=1in,rmargin=1in}
    
    

    \begin{document}
    
    
    \maketitle
    
    

    
    \section{Assignment 5: Srinithish}\label{assignment-5-srinithish}

    \subsection{Question 1}\label{question-1}

\textbf{Show how to find the maximum spanning tree of a graph, which is
the spanning tree of the largest total weight.?}

The solution for finding the maximum spanning tree as same as the
minimum spanning tree except that the edges are to be sorted in
decreasing order when chosen to form the Maximum spanning tree.

    \begin{verbatim}
MaxST(G,eW): ## ew is the edge weights
    A = {} ## empty set
    
    for each vertex v in G.V:
    ## make a disjoint set where the parents are directly retrieved    
        Make-set(v) 
    
    sort the edges of G.E into decreasing order by weight w
    
    for each edge (u,v) in G.E:  ##considered decreasing order
    
        if FIND-SET(u) != FIND-SET(v):
    
                A = A + {(u,v)} ##include in the maximum set
                Union(u,v) ## make parents of u,v point to the same
    
    return A
    
\end{verbatim}

    \subsection{Question 2}\label{question-2}

\textbf{Consider an undirected graph G = (V, E) with nonnegative edge
weights we. Suppose that you have computed a minimum spanning tree of G,
and that you have also computed shortest paths to all nodes from a
particular node s. Now suppose each edge weight is increased by 1; the
new weights are w'e = we + 1.}

\textbf{1. Does the minimum spanning tree change? Give an example where
it changes or prove it cannot change. }

    The minimum spanning tree doesnot change for the following reasons,
consider the KRUSKAL algorithm, the order of the edges condiered and
FIND-SET are the most important parts for the algorithm to execute and
form a MST

\begin{enumerate}
\def\labelenumi{\arabic{enumi}.}
\tightlist
\item
  When a constant c is added to every edge , the sorting order is not
  disturbed and hence the order remains the same as the original Graph
\item
  And the step FIND-SET is also unchanged as the parents of original
  Vertex are still the same in the new modified graph.
\end{enumerate}

Hence the MST remains the same but the Minimum Value is offset by the
c*V

    \begin{enumerate}
\def\labelenumi{\arabic{enumi}.}
\setcounter{enumi}{1}
\tightlist
\item
  \textbf{Do the shortest paths change? Give an example where they
  change or prove they cannot change?}
\end{enumerate}

Yes the shortest paths change as in below the example

\includegraphics{attachment:orig\%20graph\%20prob\%202.JPG}

    When 1 is added to each node the shortest path changes from 1
-\textgreater{} 2 \textgreater{} 3 -\textgreater{} 4 to 1
-\textgreater{} 4

    \includegraphics{attachment:Prob2\%20Chnaged\%20Graph.JPG}

    \subsection{Question 3}\label{question-3}

    The algorithm will give a Minimum Spanning Tree becuase at each
iteration we are only removing a edge if the residual graph is still
connected , in case if we had the edge in the graph still it would be
connected but with a more weight, we might as well remove the edge which
is giving additinal edge weight to the graph. Also we are only removing
the edge if the removal doenst render the trees disconnected In spanning
tree we just need to ensure that it is connected and has minimum
weights, We are keeping this invariant intact at every iteration as
weight of the tree having this edge included in the graph is higher than
if removed

    \subsection{Question 4}\label{question-4}

    Description 1. Pick a random vertex 's' and find do a BFS search
starting from this vertex 2. With this you'll find shortest segmental
distance (simple paths) to all the nodes. 3. Choose vertex whose
distance from the start vertex was the maximum say 'b' 4. Now 'b' as
start vertex do a BFS and find the vertex whose distance from b is
maximum. 5. Now distance of 'a-b' is the diameter of the Tree T

    \begin{Verbatim}[commandchars=\\\{\}]
{\color{incolor}In [{\color{incolor}0}]:} \PY{k+kn}{import} \PY{n+nn}{queue}
         \PY{c+c1}{\PYZsh{}\PYZsh{}pick a random vertex}
        
        \PY{k}{def} \PY{n+nf}{BFS}\PY{p}{(}\PY{n}{Graph}\PY{p}{,}\PY{n}{start\PYZus{}node}\PY{p}{)}\PY{p}{:}
            
            \PY{n}{Q} \PY{o}{=} \PY{n}{queue}\PY{o}{.}\PY{n}{Queue}\PY{p}{(}\PY{p}{)}
            \PY{n}{Q}\PY{o}{.}\PY{n}{put}\PY{p}{(}\PY{p}{(}\PY{n}{start\PYZus{}node}\PY{p}{,}\PY{l+m+mi}{0}\PY{p}{)}\PY{p}{)} \PY{c+c1}{\PYZsh{}\PYZsh{}node and the distance from start\PYZus{}node}
             
            \PY{n}{dictOfDistances} \PY{o}{=} \PY{p}{\PYZob{}}\PY{p}{\PYZcb{}}
                
            \PY{k}{while} \PY{n}{Q}\PY{o}{.}\PY{n}{empty}\PY{p}{(}\PY{p}{)} \PY{o}{!=} \PY{k+kc}{True}\PY{p}{:} 
                \PY{p}{(}\PY{n}{p}\PY{p}{,}\PY{n}{parent\PYZus{}dist}\PY{p}{)} \PY{o}{=} \PY{n}{Q}\PY{o}{.}\PY{n}{get}\PY{p}{(}\PY{p}{)}
                \PY{n}{p}\PY{o}{.}\PY{n}{visited} \PY{o}{=} \PY{k+kc}{True}
                
                \PY{k}{for} \PY{n}{neigh} \PY{o+ow}{in} \PY{n}{p}\PY{o}{.}\PY{n}{neighbours}\PY{p}{:}
                    
                    \PY{n}{neigh\PYZus{}distance} \PY{o}{=} \PY{n}{parent\PYZus{}dist} \PY{o}{+} \PY{l+m+mi}{1} \PY{c+c1}{\PYZsh{}\PYZsh{}one segment added}
                    
                    \PY{k}{if} \PY{n}{neigh}\PY{o}{.}\PY{n}{visited} \PY{o}{==} \PY{k+kc}{False}\PY{p}{:}
                        \PY{n}{Q}\PY{o}{.}\PY{n}{put}\PY{p}{(}\PY{p}{(}\PY{n}{neigh}\PY{p}{,}\PY{n}{neigh\PYZus{}distance}\PY{p}{)}\PY{p}{)} 
                    
                    \PY{n}{dictOfDistances}\PY{p}{[}\PY{n}{neigh}\PY{p}{]} \PY{o}{=} \PY{n}{neigh\PYZus{}distance}
            \PY{k}{return} \PY{n}{dictOfDistances}
        
        
        \PY{n}{s} \PY{o}{=} \PY{n}{random}\PY{p}{(}\PY{n}{V}\PY{p}{)}
        \PY{n}{AllDistFromS} \PY{o}{=} \PY{n}{BFS}\PY{p}{(}\PY{n}{Graph}\PY{p}{,}\PY{n}{s}\PY{p}{)} \PY{c+c1}{\PYZsh{}\PYZsh{}get the distance to all nodes from s}
        
        \PY{n}{a\PYZus{}node}\PY{p}{,}\PY{n}{distance}  \PY{o}{=} \PY{n+nb}{max}\PY{p}{(}\PY{n}{AllDistFromS}\PY{p}{,} \PY{n}{key} \PY{o}{=} \PY{n}{distFromS}\PY{p}{[}\PY{n}{i}\PY{p}{]}\PY{p}{)} \PY{c+c1}{\PYZsh{}\PYZsh{} get the node that coresponds to the max distance from s}
        
        \PY{n}{AllDistFrom\PYZus{}a} \PY{o}{=} \PY{n}{BFS}\PY{p}{(}\PY{n}{Graph}\PY{p}{,}\PY{n}{a\PYZus{}node}\PY{p}{)} \PY{c+c1}{\PYZsh{}\PYZsh{} set this node as the start node for the BFS}
        
        \PY{n}{b\PYZus{}node}\PY{p}{,}\PY{n}{Finaldistance} \PY{o}{=} \PY{n+nb}{max}\PY{p}{(}\PY{n}{AllDistFrom\PYZus{}a}\PY{p}{,} \PY{n}{key} \PY{o}{=} \PY{n}{distFromS}\PY{p}{[}\PY{n}{i}\PY{p}{]}\PY{p}{)} \PY{c+c1}{\PYZsh{}\PYZsh{} get the end node and distance from A}
\end{Verbatim}


    Above \textbf{a\_node -\/-\textgreater{} b\_node} is the max path and
the diameter is the '\textbf{Finaldistance}'

Since BFS take O(V+E) time and is called twice, Complexity is (V+E)

    \subsection{Question 5}\label{question-5}

    \includegraphics{attachment:Problem\%205.JPG}

!

    Description:

Consider the above graph Let C be a cycle in the graph which contains
edge e and e', These can be cycles in following configuration 1. Vertex
\textbf{u} connects to ** x ** then** v** connects to \textbf{y }, note
that this connection need not be direct it jus means that there is a
path 2. Vertex \textbf{u} is connected to \textbf{y} and \textbf{v}
connects to \textbf{x}

Now the algorithm,

\begin{enumerate}
\def\labelenumi{\arabic{enumi}.}
\item
  Remove edges e and e' from the graph and run dijsktra with source as
  'u' We ll have min distances d to 'y' and 'x' from 'u'
\item
  Run the dijkstra with 'v' as the source to 'x' and 'y' , we'll have
  min distances d' to x and y from v
\item
  In the first configuration when u -\/-\textgreater{} x and v
  -\/-\textgreater{} y , the total cycle weight = a = w(e) + w(e') + x.d
  + y.d'
\item
  In the first configuration when u -\/-\textgreater{}y and v
  -\/-\textgreater{} x , the total cycle weight = b = w(e) + w(e') +
  x.d' + y.d
\end{enumerate}

Then the weight of the shortest cycle containing e and e' is min(a,b)

The complexity of algorithm is equal to dijkstra \(O(V^2*E)\)

    \begin{Verbatim}[commandchars=\\\{\}]
{\color{incolor}In [{\color{incolor}0}]:} \PY{k}{def} \PY{n+nf}{findShortestCycle}\PY{p}{(}\PY{n}{Graph}\PY{p}{,}\PY{p}{(}\PY{n}{u}\PY{p}{,}\PY{n}{v}\PY{p}{)}\PY{p}{,}\PY{p}{(}\PY{n}{x}\PY{p}{,}\PY{n}{y}\PY{p}{)}\PY{p}{)}\PY{p}{:}
            
            \PY{n}{Graph}\PY{o}{.}\PY{n}{remove}\PY{p}{(}\PY{p}{(}\PY{n}{u}\PY{p}{,}\PY{n}{v}\PY{p}{)}\PY{p}{)} \PY{c+c1}{\PYZsh{}\PYZsh{} removes the edge e (u,v)}
            \PY{n}{Graph}\PY{o}{.}\PY{n}{remove}\PY{p}{(}\PY{p}{(}\PY{n}{x}\PY{p}{,}\PY{n}{y}\PY{p}{)}\PY{p}{)} \PY{c+c1}{\PYZsh{}\PYZsh{} removes the edge e\PYZsq{} (x,y)}
            
            \PY{n}{dijkstra}\PY{p}{(}\PY{n}{Graph}\PY{p}{,}\PY{n}{u}\PY{p}{)} \PY{c+c1}{\PYZsh{}\PYZsh{} find min distances from u to all vertices}
            
            \PY{n}{u\PYZus{}to\PYZus{}x} \PY{o}{=} \PY{n}{x}\PY{o}{.}\PY{n}{distance}
            \PY{n}{u\PYZus{}to\PYZus{}y} \PY{o}{=} \PY{n}{y}\PY{o}{.}\PY{n}{distance}
            
            \PY{c+c1}{\PYZsh{}\PYZsh{} initialise all d to zero }
            \PY{n}{dijkstra}\PY{p}{(}\PY{n}{Graph}\PY{p}{,}\PY{n}{v}\PY{p}{)}  \PY{c+c1}{\PYZsh{}\PYZsh{} find min distances from u to all vertices}
            
            \PY{n}{v\PYZus{}to\PYZus{}x} \PY{o}{=} \PY{n}{x}\PY{o}{.}\PY{n}{distance}
            \PY{n}{v\PYZus{}to\PYZus{}y} \PY{o}{=} \PY{n}{y}\PY{o}{.}\PY{n}{distance}
            
            
            \PY{n}{cycle\PYZus{}Weight1} \PY{o}{=} \PY{n}{Graph}\PY{o}{.}\PY{n}{weight}\PY{p}{(}\PY{p}{(}\PY{n}{u}\PY{p}{,}\PY{n}{v}\PY{p}{)}\PY{p}{)} \PY{o}{+} \PY{n}{Graph}\PY{o}{.}\PY{n}{weight}\PY{p}{(}\PY{p}{(}\PY{n}{x}\PY{p}{,}\PY{n}{y}\PY{p}{)}\PY{p}{)} \PY{o}{+} \PY{n}{u\PYZus{}to\PYZus{}x} \PY{o}{+} \PY{n}{v\PYZus{}to\PYZus{}y}
            \PY{n}{cycle\PYZus{}Weight2} \PY{o}{=} \PY{n}{Graph}\PY{o}{.}\PY{n}{weight}\PY{p}{(}\PY{p}{(}\PY{n}{u}\PY{p}{,}\PY{n}{v}\PY{p}{)}\PY{p}{)} \PY{o}{+} \PY{n}{Graph}\PY{o}{.}\PY{n}{weight}\PY{p}{(}\PY{p}{(}\PY{n}{x}\PY{p}{,}\PY{n}{y}\PY{p}{)}\PY{p}{)} \PY{o}{+} \PY{n}{v\PYZus{}to\PYZus{}x} \PY{o}{+} \PY{n}{u\PYZus{}to\PYZus{}y}
            
            
            \PY{n}{minCycleWeight} \PY{o}{=} \PY{n+nb}{min}\PY{p}{(}\PY{n}{cycle\PYZus{}Weight1}\PY{p}{,}\PY{n}{cycle\PYZus{}Weight2}\PY{p}{)}
            
            \PY{k}{if} \PY{n}{minCycleWeight} \PY{o}{!=} \PY{n+nb}{float}\PY{p}{(}\PY{l+s+s1}{\PYZsq{}}\PY{l+s+s1}{Inf}\PY{l+s+s1}{\PYZsq{}}\PY{p}{)}
                
                \PY{k}{return} \PY{n}{minCycleWeight}
            
            \PY{k}{else}\PY{p}{:} 
                \PY{k}{return} \PY{k+kc}{False}
            
                              
                              
\end{Verbatim}


    \subsection{Question 6}\label{question-6}

    \begin{enumerate}
\def\labelenumi{\alph{enumi}.}
\tightlist
\item
  In the below graph you can see that the flow is 2 + 4 (denoted in
  green) i.e 6 however the total capacity of the edges leaving A
  (source) is 8+10 i.e 18 and not equal to the flow. Hence disproved.

  \includegraphics{attachment:6a.JPG}
\end{enumerate}

    \begin{enumerate}
\def\labelenumi{\alph{enumi}.}
\setcounter{enumi}{1}
\tightlist
\item
  The statement is false. Consider the below graph, initially the
  minimum cut passes through edges (a,b) and (a,c) When all the edges
  are added with 1 the minimum cut is edge (s,a). Hence disproved

  \includegraphics{attachment:Q6b.png}
\end{enumerate}

    \subsection{Question 7}\label{question-7}

\includegraphics{attachment:Q7.png}

\begin{enumerate}
\def\labelenumi{\arabic{enumi}.}
\item
  In the graph let the people be denoted by nodes on the left in fig
  p\_nodes
\item
  Let hospitals be denoted by nodes on the right in the fig h\_nodes
\item
  Let the edges denote possibility of a person going to a certain
  hospita i.e edge (p,h) denotes hospital h is in the allowed range of
  the person p and p can visit it.
\item
  If there are hospitals h farther than allowed distance for a person p
  then there exists no edge from p to h
\item
  This problem reduces to solving bipartite graph.
\item
  Create a dummy node s as source and t as sink
\item
  Connect all the people nodes to source as shown in the figure
\item
  Connect all the nodes from hospital to sink whose capacity each is
  \textbf{n/k} and assign the same.
\item
  Assign a capacity of \textbf{1} to all the rest of the nodes
\item
  Now solve for the maximum flow using ford\_fulkerson , if the max flow
  'f' happens to be n then this assignment of people to hospitals is
  possible else its not
\end{enumerate}

The complexity of the algorithm is O(E*C)

    \begin{Verbatim}[commandchars=\\\{\}]
{\color{incolor}In [{\color{incolor}0}]:} \PY{k}{def} \PY{n+nf}{checkFeasibility}\PY{p}{(}\PY{n}{Graph}\PY{p}{,}\PY{n}{n\PYZus{}people}\PY{p}{,}\PY{n}{k\PYZus{}hospitals}\PY{p}{)}\PY{p}{:}
            \PY{n}{maxFlow} \PY{o}{=} \PY{n}{Ford\PYZus{}Fulkerson}\PY{p}{(}\PY{n}{Graph}\PY{p}{)} \PY{c+c1}{\PYZsh{}\PYZsh{}solve for max flow}
            
            \PY{k}{if} \PY{n}{maxFlow} \PY{o}{==} \PY{n}{n\PYZus{}people} \PY{p}{:}
                \PY{k}{return} \PY{k+kc}{True}
            \PY{k}{else} \PY{p}{:}
                \PY{k}{return} \PY{k+kc}{False}
\end{Verbatim}


    \subsection{Question 8}\label{question-8}

    I arrived with the below solution with help of Swarnima Sowani We can
solve this BFS assigning appropriate sets based on equaliteies and
inequalities. Say we have set1 and set2 where set 1 has all elements
with equal values and set 2 has all elements with value not equal to any
variable from set 1.

Let each of the variable \(x_1, x_2\) be vertices in the graph and let
the constrains be denoted by the edges between them. Let all equlities
be denoted as edges with weight 1 and inequalities be denoted with
weight -1

say if \(x_1 = x_2\) then edgeweight of \((x_1,x_2)\) is 1

say if \(x_1 \ne x_2\) then edgeweight of \((x_1,x_2)\) is -1

Starting with one node and applying BFS, we will explore all the
neighbors and assign same set for equal variables and opposite sets for
unequal variables. If some variable has already been assigned a set
which is not following the constraint then return false else return True

Example: For example in question, We will start with node x1 and assign
it to set1 and insert it to queue. While queue not empty: Take the
element in queue which is x1, and check all its neighbors.

Since x1=x2 and x2.set = None, x2.set = x1.set that is both x1 and x2
are assigned to set1 Then x1 != x4 and x4.set ==null so, x4 is assigned
to set opposite to x1 that is set2. There are no more neighbors of x1 so
start with next variable in queue that is x2. x2 has neighbor x3 with
equality constraint and x3 has not assigned any set. So, set of x3 will
be same as x2. x2 has no more neighbors so we will move with next
element in queue that is x3. x3 has only one neighbor x3=x4 but x4 is
assigned to set2 which is not same as the set of x3. Hence the
constraints are violated.

    \begin{Verbatim}[commandchars=\\\{\}]
{\color{incolor}In [{\color{incolor}0}]:} \PY{k}{def} \PY{n+nf}{isSatisifying}\PY{p}{(}\PY{n}{Graph}\PY{p}{)}\PY{p}{:}
            
            \PY{n}{Q} \PY{o}{=} \PY{n}{queue}\PY{o}{.}\PY{n}{Queue}\PY{p}{(}\PY{p}{)}
            \PY{n}{start\PYZus{}node} \PY{o}{=} \PY{n}{random}\PY{p}{(}\PY{n}{Graph}\PY{o}{.}\PY{n}{Vertices}\PY{p}{)}
            \PY{n}{Q}\PY{o}{.}\PY{n}{put}\PY{p}{(}\PY{p}{(}\PY{n}{start\PYZus{}node}\PY{p}{,}\PY{l+m+mi}{0}\PY{p}{)}\PY{p}{)} \PY{c+c1}{\PYZsh{}\PYZsh{}node and the distance from start\PYZus{}node}
            
            \PY{c+c1}{\PYZsh{}\PYZsh{}initialise all vertex set to None}
            \PY{k}{for} \PY{n}{v} \PY{o+ow}{in} \PY{n}{Graph}\PY{o}{.}\PY{n}{Vertices}\PY{p}{:}
                \PY{n}{v}\PY{o}{.}\PY{n}{set} \PY{o}{=} \PY{k+kc}{None}
            
            \PY{n}{start\PYZus{}node}\PY{o}{.}\PY{n}{set} \PY{o}{=} \PY{l+m+mi}{1}
            
            \PY{k}{while} \PY{n}{Q}\PY{o}{.}\PY{n}{empty}\PY{p}{(}\PY{p}{)} \PY{o}{!=} \PY{k+kc}{True}\PY{p}{:} 
                \PY{p}{(}\PY{n}{p}\PY{p}{,}\PY{n}{parent\PYZus{}dist}\PY{p}{)} \PY{o}{=} \PY{n}{Q}\PY{o}{.}\PY{n}{get}\PY{p}{(}\PY{p}{)}
                \PY{n}{p}\PY{o}{.}\PY{n}{visited} \PY{o}{=} \PY{k+kc}{True}
                
                
                \PY{k}{for} \PY{n}{neigh} \PY{o+ow}{in} \PY{n}{p}\PY{o}{.}\PY{n}{neighbours}\PY{p}{:}
                    
                    \PY{n}{sign} \PY{o}{=} \PY{n}{Graph}\PY{o}{.}\PY{n}{weights}\PY{p}{(}\PY{p}{(}\PY{n}{p}\PY{p}{,}\PY{n}{neigh}\PY{p}{)}\PY{p}{)}
                    
                    \PY{k}{if} \PY{n}{neigh}\PY{o}{.}\PY{n}{set} \PY{o}{==} \PY{k+kc}{None}\PY{p}{:} \PY{c+c1}{\PYZsh{}\PYZsh{}if not already assigned the set}
                        \PY{k}{if} \PY{n}{sign} \PY{o}{==} \PY{o}{\PYZhy{}}\PY{l+m+mi}{1}\PY{p}{:}  \PY{c+c1}{\PYZsh{}\PYZsh{}if its inequality}
                            \PY{n}{neigh}\PY{o}{.}\PY{n}{set} \PY{o}{=} \PY{o}{\PYZhy{}}\PY{l+m+mi}{1}\PY{o}{*}\PY{n}{p}\PY{o}{.}\PY{n}{set}  \PY{c+c1}{\PYZsh{}\PYZsh{}give the opposite set}
                        \PY{k}{if} \PY{n}{sign} \PY{o}{==} \PY{l+m+mi}{1}\PY{p}{:}
                            \PY{n}{neigh}\PY{o}{.}\PY{n}{set} \PY{o}{=} \PY{n}{p}\PY{o}{.}\PY{n}{set} \PY{c+c1}{\PYZsh{}\PYZsh{}else same set}
                            
                            
                    \PY{k}{if} \PY{n}{neigh}\PY{o}{.}\PY{n}{set} \PY{o}{!=} \PY{k+kc}{None}\PY{p}{:} \PY{c+c1}{\PYZsh{}\PYZsh{}if already a set is assigned}
                        
                        \PY{k}{if} \PY{n}{sign} \PY{o}{==} \PY{l+m+mi}{1}\PY{p}{:} \PY{c+c1}{\PYZsh{}\PYZsh{}if its equality}
                            
                            \PY{k}{if} \PY{n}{neigh}\PY{o}{.}\PY{n}{set} \PY{o}{!=} \PY{n}{p}\PY{o}{.}\PY{n}{set}\PY{p}{:} \PY{c+c1}{\PYZsh{}\PYZsh{}if its parent and neigh do not agree}
                                \PY{k}{return} \PY{k+kc}{False}
                        \PY{k}{if} \PY{n}{sign} \PY{o}{==} \PY{o}{\PYZhy{}}\PY{l+m+mi}{1}\PY{p}{:}
                            
                            \PY{k}{if} \PY{n}{neigh}\PY{o}{.}\PY{n}{set} \PY{o}{==} \PY{n}{p}\PY{o}{.}\PY{n}{set}\PY{p}{:}
                                \PY{k}{return} \PY{k+kc}{False}
                                
                    \PY{k}{if} \PY{n}{neigh}\PY{o}{.}\PY{n}{visited} \PY{o}{==} \PY{k+kc}{False}\PY{p}{:}
                        \PY{n}{Q}\PY{o}{.}\PY{n}{put}\PY{p}{(}\PY{p}{(}\PY{n}{neigh}\PY{p}{,}\PY{n}{neigh\PYZus{}distance}\PY{p}{)}\PY{p}{)} 
                    
            
\end{Verbatim}



    % Add a bibliography block to the postdoc
    
    
    
    \end{document}
